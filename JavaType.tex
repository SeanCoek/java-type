\documentclass{paper}
\usepackage{amsmath}
\usepackage{amsfonts}

%
\usepackage{paralist}
\usepackage{float}
\usepackage{url}
\usepackage{hyperref}
\usepackage[capitalise]{cleveref}
\usepackage[utf8]{inputenc}
\usepackage[autostyle=false, style=english]{csquotes}
\usepackage{listings}
\usepackage[justification=centering]{caption} % centre-align listings captions
\usepackage{lstlinebgrd}
\usepackage{graphicx}
\usepackage{ucs}
\usepackage{fixltx2e}
\usepackage{subcaption}
\usepackage{fancyvrb}
\usepackage{xspace}
%\usepackage{fullpage}
\usepackage{stmaryrd}


% Force all figures & tables to use bottom captions
%\floatstyle{plain}
%\restylefloat{figure}
%\restylefloat{table}

% Make all figures and tables absolutely-positioned by default
%\floatplacement{figure}{H}
%\floatplacement{table}{H}

% !! Hacky solution !! Reduce the size of verbatim font to prevent overflow
% \fvset{fontsize=\tiny}

%\MakeOuterQuote{"}

\hypersetup{%
	colorlinks,
	citecolor=black,
	filecolor=black,
	linkcolor=black,
	urlcolor=black
}

\newcommand{\todo}[1]{%
\begin{center}%
  \fbox{%
    \begin{minipage}{0.95\columnwidth}\small
      \textcolor{red}{\textbf{TODO:} #1}
    \end{minipage}%
  }
\end{center}
}
\newcommand{\keyword}[1]{\mathsf{#1}}
\newcommand{\kwboolean}{\keyword{boolean}}
\newcommand{\kwint}{\keyword{int}}
\newcommand{\kwfloat}{\keyword{float}}
\newcommand{\kwdouble}{\keyword{double}}
\newcommand{\kwpublic}{\keyword{public}}
\newcommand{\kwprivate}{\keyword{private}}
\newcommand{\kwprotected}{\keyword{protected}}
\newcommand{\kwstatic}{\keyword{static}}
\newcommand{\kwfinal}{\keyword{final}}
\newcommand{\kwnull}[0]{\keyword{null}}
\newcommand{\kwnew}[0]{\keyword{new}}
\newcommand{\kwextends}[0]{\keyword{extends}}
\newcommand{\kwclass}[0]{\keyword{class}}
\newcommand{\kwthis}[0]{\keyword{this}}
\newcommand{\kwsuper}[0]{\keyword{super}}
\newcommand{\kwif}[0]{\keyword{if}}
\newcommand{\kwthen}[0]{\keyword{then}}
\newcommand{\kwelse}[0]{\keyword{else}}
\newcommand{\kwskip}[0]{\keyword{skip}}
\newcommand{\kwfielddef}{\keyword{field\mbox{-}def}}
\newcommand{\kwmethoddef}{\keyword{method\mbox{-}def}}
\newcommand{\kwarglist}{\keyword{arg\mbox{-}list}}
\newcommand{\kwloc}[1]{\keyword{loc}(#1)}

\newcommand{\tuple}[1]{\langle #1 \rangle}
\newcommand{\suc}{\texttt{succ}}
\newcommand{\nxt}{\texttt{next}}
\newcommand{\LT}{\longrightarrow}
\newcommand{\set}[1]{\{#1\}}
\newcommand{\sem}[1]{\llbracket #1\rrbracket}
\newcommand{\T}{\mathcal{T}}

%%%%%% NOTE: Page 15-page limit for SAS'18 (excl. bib+app) %%%%%%

%
\begin{document}
\title{A Simple Type-Based Points-to Analysis}
\author{\today}

\maketitle

We take a simplified syntax from the website~\cite{amoeller-web} for the Java Programming language, specified in EBNF.

\begin{figure}[!htbp]\centering
	\begin{tabular}[c]{lll} 
        $prog$&$::=$&$C^*$\\
		$C$&$::=$&$\kwclass\ A\ [\kwextends\ B] \ \{\kwfielddef;\ \kwmethoddef\}$\\
        $\kwfielddef$&$::=$&$f{=}e\ |\ \kwfielddef; \kwfielddef$\\
		$\kwmethoddef$&$::=$&$m(x) \ \{s\}\ |\ \kwmethoddef; \kwmethoddef$\\
		$s$&$::=$&$\ \epsilon |\  x {=} e \ |\ x.f{=}x \ |\ x.m(\kwarglist)$ \\
		&&$|\ \kwif \ x \ \kwthen \ s \ \kwelse \ s \ |\ s;s$\\
		$e$&$::=$&$c \ |\ x \ |\ x.f\ |\ \kwnew \ A(\kwarglist)\ |\ x.m(\kwarglist)$\\
        $\kwarglist$&$::=$&$e\ |\ e,\kwarglist$\\
        $c$&$::=$&$n$ where $n\in\mathbb{R}$\\
	\end{tabular}
	\caption{The starting syntax of Java. \label{fig:syntax}}
\end{figure}

For this simplified Java syntax, we omit some of the basic Java syntax, such as basic types (e.g., $\kwboolean$, $\kwint$, $\kwdouble$), constructor and field modifiers (e.g., $\kwpublic$, $\kwprivate$, $\kwstatic$, $\kwfinal$), the interface structure, exception handling, and specialized keywords (e.g., $\kwthis$, $\kwsuper$). We assume the ways of handling such structures can be added straightforwardly on this minimized syntax. Moreover, we assume that a given program has always correct syntax, otherwise it will not pass compilation. We assume values are within ranges of their defined variables, and we ignore the cases of overflows.

%%% objects are location based, points-to algrithm Andersen-styled, filed sensitive and object sensitive (not flow- or context-sensitive)




\begin{thebibliography}{4}

\bibitem{amoeller-web} 
A. M$\o$ller.
https://users-cs.au.dk/amoeller/RegAut/JavaBNF.html
 
\bibitem{andersen-94}
L. O. Andersen. 
Program Analysis and Specialization for the C Programming Language. 
PhD thesis, University of Copenhagen, DIKU, 1994.

\end{thebibliography}
\end{document}
